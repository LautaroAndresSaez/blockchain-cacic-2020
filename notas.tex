\documentclass[]{book}
\usepackage[utf8]{inputenc}
\usepackage[spanish,es-tabla]{babel}
\usepackage{amsmath}
\usepackage{amsfonts}
\usepackage{amssymb}
\usepackage{graphicx}



\title{Desarrollo de aplicaciones en la Blockchain \\ CACIC 2020}
\author{Saez, Lautaro Andres}
\date{}

\begin{document}
    \maketitle
    \tableofcontents
    \chapter{ Introducción a la Blockchain }

    Es una tematica nueva, no es posible saber todo lo que esta pasando. Esto se 
    debe a que es un area en desarrollo y en constante crecimiento.

    Como es un tema muy nuevo cuando encontramos un problema de algun ejemplo 
    debemos ver las librerias, ya que no mantienen compatibilidad con versiones anteriores.

    \section{Que es la Blockchain?}

    La Blockchain es: 

    \begin{itemize}
        \item Base de datos distribuida
        \item Libro de registro de operaciones
        \item Simil a un libro mayor (contabilidad)
    \end{itemize}

    Si sucede un error este no puede ser modificado, si no que debo generar una nueva entrada 
    que corrija dicho problema.

    \section{Por que es importante?}

    La Blockchain es importante ya que elimina cualquier intermediario.
    Es la unica base de datos distribuida que no pudo ser hackeada desde 2009.

    Permite complementar o sustituir los sistemas clasicos de compra/venta.

    \section{Aspectos de la Blockchain}

    Es una red distribuida. La conexion se estable por P2P como por ejemplo torrent. 
    Es importante saber que la informacion esta encriptada, tenemos la seguridad que la informacion 
    que esta en la Blockchain (si la podemos verificar) entonces esa informacion es la que guardo el usuario.

    A la informacion que viaja se la denomina \textbf{tokens}, esto es muy importante para cuando 
    trabajemos los criptoactivos, como el bitcoin.

    \section{Como funciona el Blockchain?}

    Si existen 2 usuarios llamemos A y B. Si A quiere realizar una transaccion a B
    entonces A debe generar un bloque (lo que representa una transaccion), luego el 
    bloque se trasmite el bloque a toda la red y se busca un HASH a fuerza bruta 
    (mineria). Una vez que alguna de las maquinas encuentra el HASH 
    el resto de los nodos valida dicho valor para verificar que el bloque es valido.
    A los bloques se les agrega el HASH de la caja anterior, armando una cadena. Finalmente 
    B ve reflajada la transaccion en su billetera.

    \section{Mineria}

    Es una sucesion de calculos complejos. El primer minero que selle el bloque 
    recive una paga en dicha criptomoneda, para este caso tomaremos el bitcoin.
    El bloque queda registrado de forma permanente en la cadena de bloques.
    
    Añadir un bloque nuevo se vuelve cada vez mas complejo.

    \section{Smart Contract}

    Es un programa que facilita, asegura, hace cumplir y ejecuta acuerdos entre 2 partes, como 
    lo hacen los bancos, los estudios, etc.

    

    \chapter{ Introducción a los cripto valores }
    \chapter{ Bitcoin, Altercoins y Stablecoins }
    \chapter{ Distintos tipos de Blockchain }
    \chapter{ Protocolos de consenso }
    \chapter{ Billeters Digitales }
    \chapter{ Plataforma de desarrollo basada en Ethereum }
    \chapter{ Programación en Solidity }
    \chapter{ Seguridad en aplicaciones basadas en Blockchain }
    \chapter{ Aplicaciones Financieras en Blockchain }
    \chapter{ Gestión de proyectos en la Blockchain }
    \chapter{ Introducción a los medios de pago basados en Ripple }
    \chapter{ Computación Cuántica y Blockchain }
    \chapter{ Integración entre Blockchain e Inteligencia Artificial }
    \chapter{ Integración entre Blockchain e Internet of Things }
    \chapter{ Integración de todos los aspectos analizados }
\end{document}