\documentclass[]{book}
\usepackage[utf8]{inputenc}
\usepackage[spanish,es-tabla]{babel}
\usepackage{amsmath}
\usepackage{amsfonts}
\usepackage{amssymb}
\usepackage{graphicx}

\title{Desarrollo de aplicaciones en la Blockchain \\ CACIC 2020}
\author{Saez, Lautaro Andres}
\date{}

\begin{document}
    \maketitle
    \tableofcontents
    \chapter{ Introducción a la Blockchain }

    Es una tematica nueva, no es posible saber todo lo que esta pasando. Esto se 
    debe a que es un area en desarrollo y en constante crecimiento.

    Como es un tema muy nuevo cuando encontramos un problema de algun ejemplo 
    debemos ver las librerias, ya que no mantienen compatibilidad con versiones anteriores.

    \section{Que es la Blockchain?}

    La Blockchain es: 

    \begin{itemize}
        \item Base de datos distribuida
        \item Libro de registro de operaciones
        \item Simil a un libro mayor (contabilidad)
    \end{itemize}

    Si sucede un error este no puede ser modificado, si no que debo generar una nueva entrada 
    que corrija dicho problema.

    \chapter{ Introducción a los cripto valores }
    \chapter{ Bitcoin, Altercoins y Stablecoins }
    \chapter{ Distintos tipos de Blockchain }
    \chapter{ Protocolos de consenso }
    \chapter{ Billeters Digitales }
    \chapter{ Plataforma de desarrollo basada en Ethereum }
    \chapter{ Programación en Solidity }
    \chapter{ Seguridad en aplicaciones basadas en Blockchain }
    \chapter{ Aplicaciones Financieras en Blockchain }
    \chapter{ Gestión de proyectos en la Blockchain }
    \chapter{ Introducción a los medios de pago basados en Ripple }
    \chapter{ Computación Cuántica y Blockchain }
    \chapter{ Integración entre Blockchain e Inteligencia Artificial }
    \chapter{ Integración entre Blockchain e Internet of Things }
    \chapter{ Integración de todos los aspectos analizados }
\end{document}